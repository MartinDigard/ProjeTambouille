\documentclass{article}
\usepackage{ae,lmodern} % Polices vectorielles
\usepackage[french]{babel} % Traitement du français
\usepackage[utf8]{inputenc} % encodage d’entrée
\usepackage[T1]{fontenc} % encodage de sortie
\usepackage{verbatim} % inclure du code source
\usepackage{graphicx} % inclure des images
\usepackage{hyperref} % inclure des adresses web
\newcommand\tab[1][5mm]{\hspace*{#1}}
\parindent=0em % Supprimer les marges en début de paragraphes.


\begin{document}
	\title{Titre du projet}
	\author{Martin Digard}
	\date{}
	\maketitle
	\chapter*{Chapitre}
	Texte ?
	\section*{Section}
	Texte \\	
	\subsection*{Sub-section}
	Texte
	\begin{verbatim}
	print("pour écrire du code utiliser verbatim")
	\end{verbatim}
	Pour afficher une image :
	%\includegraphics[height=50mm, width=110mm]{imgs/image.png} \\
	\newpage{} % Nouvelle page
	$\rightarrow$ Faire une flèche
	$\Rightarrow$ Faire une flèche double
        note de bas de page\footnote{voilà}
        \url{https://www.martindigard.com/}
        \tab c’est une tabulation
\end{document}
