\usepackage[utf8]{inputenc}
\usepackage[T1]{fontenc}
\usepackage{lipsum} % juste utile ici pour générer du faux texte}
\usepackage{amsmath,amsfonts,amssymb} %extensions de l'ams pour les mathématiques
\usepackage{graphicx} % pour insérer images et pdf entre autres
\usepackage[left=3.5cm,right=2cm,top=2cm,bottom=2.5cm]{geometry}%réglages des marges du document selon vos préférences ou celles de votre établissement
\usepackage[french]{babel}%pour un document en français
\usepackage{listings}%pour insérer du code source
\usepackage{hyperref}%rend actif les liens, références croisées, toc…

\usepackage{fancyhdr}%pour les en-têtes et pieds de pages
\setlength{\headheight}{14.2pt}% hauteur de l'en-tête
%%%%%%%%%%%%%%%%%%%style front%%%%%%%%%%%%%%%%%%%%%%%%%%%%%%%%%%%%%%%%% 
\fancypagestyle{front}{%
	\fancyhf{}%on vide l'en-tête
	\fancyfoot[C]{page \thepage}%
	\renewcommand{\headrulewidth}{0pt}%trait horizontal pour l'en-tête
	\renewcommand{\footrulewidth}{0.4pt}%trait horizontal pour le pied de page
}
%\renewcommand{\chaptermark}[1]{%
%\markboth{\MakeUppercase{%
%\chaptername}\ \thechapter.%
%\ #1}{}}
%%%%%%%%%%%%%%%%%%%style main%%%%%%%%%%%%%%%%%%%%%%%%%%%%%%%%%%%%
\fancypagestyle{main}{%
	\fancyhf{}
	\renewcommand{\chaptermark}[1]{\markboth{\chaptername\ \thechapter.\ ##1}{}}% redéfinition pour avoir ici les titres des chapitres des sections en minuscules
	\renewcommand{\sectionmark}[1]{\markright{\thesection\ ##1}}
	\fancyhead[c]{}
	\fancyhead[RO,LE]{\rightmark}%
	\fancyhead[LO,RE]{\leftmark}
	\fancyfoot[C]{}
	\fancyfoot[RO,LE]{page \thepage}%
	\fancyfoot[LO,RE]{Mon rapport}
}
%%%%%%%%%%%%%%%%%%%style back%%%%%%%%%%%%%%%%%%%%%%%%%%%%%%%%%%%%%%%%%  
\fancypagestyle{back}{%
	\fancyhf{}%on vide l'en-tête
	\fancyfoot[C]{page \thepage}%
	\renewcommand{\headrulewidth}{0pt}%trait horizontal pour l'en-tête
	\renewcommand{\footrulewidth}{0.4pt}%trait horizontal pour le pied de pages
}